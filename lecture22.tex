\section{Lecture 22: \underline{Black Holes}}

Only depends on Lectures 1-15, so does lecture on ``Wednesday''

Schwarzschild solution also vacuum solution (from tutorial EY : oh no, must do tutorial)

Study the Schwarzschild as a vacuum solution of the Einstein equation:

$m = G_N M$ where $M$ is the ``mass''
\[
g = \left( 1 - \frac{2m}{r} \right) dt \otimes dt - \frac{1}{ 1 - \frac{2m}{r} } dr \otimes dr - r^2 ( d\theta \otimes d\theta + \sin^2{\theta} d\varphi \otimes d\varphi
\]
in the so-called \underline{Schwarzschild coordinates}  $\begin{aligned} & & & & \quad \\
 t \quad & r \quad & \theta \quad & \varphi \\ 
 (-\infty,\infty) \quad & (0,\infty) \quad & (0,\pi) \quad & (0,2\pi) \end{aligned}$

What staring at this metric for a while, two questions naturally pose themselves:

\begin{enumerate}
\item[(i)] What exactly happens \@ $r= 2m$?

$\begin{aligned} & & & & \quad \\
 t \quad & r \quad & \theta \quad & \varphi \\ 
 (-\infty,\infty) \quad & (0,2m) \cup ( 2m, \infty) \quad & (0,\pi) \quad & (0,2\pi) \end{aligned}$


\item[(ii)] Is there anything (in the real world) beyond $\begin{aligned} & \quad \\
  & t \to -\infty \\
  & t\to +\infty \end{aligned}$?

\underline{idea}: Map of Linz, blown up

Insight into these two issues is afforded by stopping to stare.  

Look at \emph{geodesic} of $g$, instead.

\end{enumerate}

\subsection{Radial null geodesics}

null - $g(v_{\gamma},v_{\gamma} ) = 0$

Consider null geodesic in ``\underline{Schd}''

\[
S[\gamma ] = \int d\lambda \left[ \left( 1 - \frac{2m}{r} \right)\dot{t}^2 - \left(1 - \frac{2m}{r} \right)^{-1} \dot{r}^2 - r^2( \dot{\theta}^2 + \sin^2{\theta} \dot{\varphi}^2 ) \right]
\]
with $[\dots ] =0$

and one has, in particular, the $t$-eqn. of motion:

\[
\left( \left( 1-  \frac{2m}{r} \right) \dot{t} \right)^{.} = 0
\]
$\Longrightarrow$
\[
\boxed{ \left( 1 - \frac{2m}{r} \right)\dot{t} = k } = \text{ const. }
\]
Consider \underline{radial} null geodesics \\
$\theta \overset{!}{=} \text{ const. }$ \quad \quad \, $\varphi = \text{ const. }$

From $\Box $ and $\Box $
\[
\Longrightarrow \dot{r}^2 = k^2 \leftrightarrow \dot{r} = \pm k
\]
\[
\Longrightarrow r(\lambda) = \pm k \cdot \lambda
\]
Hence, we may consider 
\[
\widetilde{t}(r) := t(\pm k\lambda)
\]

\underline{Case A:} $\oplus$

\[
\frac{d\widetilde{t}}{dr} = \frac{ \dot{ \widetilde{t}} }{ \dot{r}} = \frac{k}{ \left( 1 - \frac{2m}{r} \right) k } = \frac{r}{r-2m}
\]
\[
\Longrightarrow \widetilde{t}_+(r) = r + 2m \ln{ |r-2m | }
\]
(\textbf{outgoing} null geodesics)

\underline{Case b.} $\pm$ (Circle around $-$, consider $-$):

\[
\widetilde{t}_-(r) = -r - 2m \ln{ |r - 2m | }
\]
(\textbf{ingoing} null geodesics)

Picture

\subsection{Eddington-Finkelstein}

Brilliantly simple idea: 

change (on the domain of the Schwarzschild coordinates) to different coordinates, s.t.  \\ 
in those new coordinates, \\
\emph{ingoing} null geodesics appear as straight lines, of slope $-1$ 

This is achieved by 

\[
\bar{t}(t,r,\theta, \varphi) := t + 2m \ln{ | r-2m | }
\]
\underline{Recall}: ingoing null geodesic has 
\[
\widetilde{t}(r) = -(r + 2m \ln{ |r-2m |} )  \quad \quad \, (Schd coords)
\]

\[
\Longleftrightarrow \bar{t} - 2m \ln{ |r-2m |} = -r - 2m \ln{ |r-2m |} + \text{ const. }
\]
\[
\therefore \bar{t} = -r + \text{ const. }
\]

(Picture)

\emph{outgoing} null geodesics

\[
\bar{t} = r + 4 m \ln{ |r - 2m| } + \text{ const. }
\]

Consider the new chart $(V,g)$ while $(U,x)$ was the Schd chart.

\[
\underbrace{U}_{\text{Schd}} \bigcup \lbrace \text{ horizon } \rbrace = V
\]
``chart image of the horizon''

Now calculate the \emph{Schd metric $g$ } w.r.t. Eddington-Finkelstein coords.

\[
\begin{aligned}
  & \bar{t}(t,r,\theta,\varphi) = t + 2m\ln{ |r -2m | } \\
  & \bar{r}(t,r,\theta,\varphi) = r \\
  & \bar{\theta}(t,r,\theta,\varphi) = \theta \\
  & \bar{\varphi}(t,r,\theta,\varphi) = \varphi
\end{aligned}
\]

EY : 20150422 I would suggest that after seeing this, one would calculate the metric by your favorite CAS.  I like the Sage Manifolds package for Sage Math.  

\href{https://github.com/ernestyalumni/diffgeo-by-sagemnfd/blob/master/Schwarzschild_BH.sage}{Schwarzschild\_BH.sage on github}

\href{https://www.patreon.com/file?s=645287&h=2254352&i=108637}{Schwarzschild\_BH.sage on Patreon}

\href{https://drive.google.com/file/d/0B1H1Ygkr4EWJdllTR3czQU9DeW8/view?usp=sharing}{Schwarzschild\_BH.sage on Google Drive}

\lstset{language=Python,basicstyle=\scriptsize\ttfamily,
    commentstyle=\ttfamily\color{gray}}
\begin{lstlisting}[frame=single]
sage: load(``Schwarzschild_BH.sage'')
4-dimensional manifold 'M'
  expr = expr.simplify_radical()
Levi-Civita connection 'nabla_g' associated with the Lorentzian metric 'g' on the 4-dimensional manifold 'M'
Launched png viewer for Graphics object consisting of 4 graphics primitives
\end{lstlisting}

Then calculate the Schwarzschild metric $g$ but in Eddington-Finkelstein coordinates.  Keep in mind to calculate the set of coordinates that uses $\bar{t}$, not $\widetilde{t}$: 

\begin{lstlisting}[frame=single]
sage: gI.display()
gI = (2*m - r)/r dt*dt - r/(2*m - r) dr*dr + r^2 dth*dth + r^2*sin(th)^2 dph*dph
sage: gI.display( X_EF_I_null.frame())
gI = (2*m - r)/r dtbar*dtbar + 2*m/r dtbar*dr + 2*m/r dr*dtbar + (2*m + r)/r dr*dr + r^2 dth*dth + r^2*sin(th)^2 dph*dph
\end{lstlisting}



