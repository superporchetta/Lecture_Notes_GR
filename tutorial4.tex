\section*{Tutorial 4 Differentiable Manifolds}

EY : 20151109 The \url{gravity-and-light.org} website, where you can download the tutorial sheets \emph{and} the full length videos for the tutorials and lectures, are no longer there.  $=($  

Hopefully, the YouTube video will remain: \url{https://youtu.be/FXPdKxOq1KA?list=PLFeEvEPtX_0RQ1ys-7VIsKlBWz7RX-FaL}

\exercisehead{1: True or false?} \emph{These basic questions are designed to spark discussion and as a self-test.}

Tick the correct statements, but not the incorrect ones!

\begin{enumerate}
  \item[(a)] The function $f: \mathbb{R} \to \mathbb{R}$, \dots
    \begin{itemize}
      \item  
      \item
      \item \dots , defined by $f(x) = |x^3|$, lies in $C^3(\mathbb{R} \to \mathbb{R})$.  

\solutionhead{1a3} For $f: \mathbb{R} \to \mathbb{R}$, $f(x) = |x^3| = \begin{cases} x^3 & \text{ if } x \geq 0 \\
  -x^3 & \text{ if } x < 0 \end{cases}$ 
\[
\begin{aligned}
  & f'(x) = \begin{cases} 3x^2  & \text{ if } x \geq 0 \\
    -3x^2 & \text{ if } x < 0 \end{cases} \\ 
  & f''(x) = \begin{cases} 6x  & \text{ if } x \geq 0 \\
    -6x & \text{ if } x < 0 \end{cases} 
\end{aligned}
\]
Thus, 
\[
\boxed{ f(x) = |x^3| \in C^1(\mathbb{R}) \text{ but } f(x) \notin C^2(\mathbb{R}) \subseteq C^3(\mathbb{R}) }
\]
      \item
      \item
\end{itemize}
  \item[(b)]
  \item[(c)]
\end{enumerate}

\textbf{Short} \exercisehead{4: Undergraduate multi-dimensional analysis }

\emph{A good notation and basic results for partial differentiation}.

For a map $f: \mathbb{R}^d \to \mathbb{R}$ we denote by the map $\partial_i f: \mathbb{R}^d \to \mathbb{R}$ the partial derivative with respect to the $i$-th entry.

\questionhead{:} Given a function
\[
f: \mathbb{R}^3 \to \mathbb{R}; \, (\alpha, \beta, \delta) \mapsto f(\alpha,\beta,\delta) := \alpha^3\beta^2 + \beta^2 \delta + \delta
\]
calculate the values of the following derivatives:

\solutionhead{:}

\begin{itemize}
  \item $(\partial_2f)(x,y,z) = $
  \item $(\partial_1f)(\square,\circ,*) =$
  \item $(\partial_1 \partial_2 f)(a,b,c) = $ 
  \item $(\partial_3^2 f)(299,1222,0) =$
\end{itemize}

EY: 20151110

For $f(\alpha,\beta,\delta) := \alpha^3\beta^2 + \beta^2 \delta + \delta$, or $f(x,y,z) = x^3 y^2 + y^2 z + z$, 
\[
\begin{aligned}
  & (\partial_2 f) = 2(x^3y+yz) \\ 
  & (\partial_1 f) = 3x^2 y^2 \\ 
  & (\partial_1\partial_2 f) = 6x^2 y \\ 
  & (\partial_3^2f) = 0 
\end{aligned}
\]
and so 
\begin{itemize}
  \item $(\partial_2f)(x,y,z) = 2(x^3 y + yz)  $
  \item $(\partial_1f)(\square,\circ,*) = 3\square^2 \circ^2$
  \item $(\partial_1 \partial_2 f)(a,b,c) = 6a^2 b$ 
  \item $(\partial_3^2 f)(299,1222,0) = 0$
\end{itemize}



\exercisehead{5: Differentiability on a manifold}

\emph{How to deal with functions and curves in a chart} 

Let $(M, \mathcal{O}, \mathcal{A})$ be a smooth $d$-dimensional manifold.  Consider a chart $(U,x)$ of the atlas $\mathcal{A}$ together with a smooth curve $\gamma : \mathbb{R} \to U$ and a smooth function $f:U \to \mathbb{R}$ on the domain $U$ of the chart. 

\questionhead{:} Draw a commutative diagram containing the chart domain, chart map, function, curveand the respective representatives of the function and the curve in the chart. 

\solutionhead{:}

\begin{tikzpicture}[decoration=snake]
  \matrix (m) [matrix of math nodes, row sep=4em, column sep=6em, minimum width=2em]
  {
 \mathbb{R} & U & \mathbb{R}^d \\
& \mathbb{R} &  \\
};
  \path[->]
  (m-1-1) edge node [above] {$\gamma$} (m-1-2)
          edge [bend left=40] node [auto] {$x\circ \gamma$} (m-1-3)
  (m-1-3) edge [bend left=15] node [auto] {$x^{-1}$} (m-1-2)
          edge node [right] {$(f\circ x^{-1})$ } (m-2-2)
  (m-1-2) edge node [left] {$f$} (m-2-2)
          edge node [auto] {$x$} (m-1-3);
\end{tikzpicture} \quad \quad \, \begin{tikzpicture}[decoration=snake]
  \matrix (m) [matrix of math nodes, row sep=4em, column sep=6em, minimum width=2em]
  {
 \tau \in \mathbb{R} & p \in U & x(p) = (x\circ \gamma)(\tau) \in \mathbb{R}^d \\
& f(p) \in \mathbb{R} &  \\
};
  \path[|->]
  (m-1-1) edge node [above] {$\gamma$} (m-1-2)
          edge [bend left=40] node [auto] {$x\circ \gamma$} (m-1-3)
  (m-1-3) edge [bend left=15] node [auto] {$x^{-1}$} (m-1-2)
          edge node [right] {$(f\circ x^{-1})$ } (m-2-2)
  (m-1-2) edge node [left] {$f$} (m-2-2)
          edge node [auto] {$x$} (m-1-3);
\end{tikzpicture}



\questionhead{:} Consider, for $d=2$,
\[
(x\circ \gamma)(\lambda):= (\cos{(\lambda)}, \sin{(\lambda)} ) \text{ and } (f\circ x^{-1})((x,y)) := x^2 +y^2
\]
Using the chain rule, calculate
\[
(f\circ \gamma)'(\lambda)
\]
explicitly.

\solutionhead{:}

EY : 20151109 Indeed, the domains and codomains of this $f\gamma$ mapping makes sense, from $\mathbb{R} \to \mathbb{R}$ for 
\begin{tikzpicture}[decoration=snake]
  \matrix (m) [matrix of math nodes, row sep=4em, column sep=6em, minimum width=2em]
  {
 \mathbb{R} & U & \mathbb{R}^d \\
& \mathbb{R} &  \\
};
  \path[->]
  (m-1-1) edge node [above] {$\gamma$} (m-1-2)
          edge [bend left=40] node [auto] {$x\circ \gamma$} (m-1-3)
          edge node [auto] {$f\circ \gamma$} (m-2-2)
  (m-1-3) edge [bend left=15] node [auto] {$x^{-1}$} (m-1-2)
          edge node [right] {$(f\circ x^{-1})$ } (m-2-2)
  (m-1-2) edge node [left] {$f$} (m-2-2)
          edge node [auto] {$x$} (m-1-3);
\end{tikzpicture} \quad \quad \, \begin{tikzpicture}[decoration=snake]
  \matrix (m) [matrix of math nodes, row sep=4em, column sep=6em, minimum width=2em]
  {
 \tau \in \mathbb{R} & p \in U & x(p) = (x\circ \gamma)(\tau) \in \mathbb{R}^d \\
& f(p) \in \mathbb{R} &  \\
};
  \path[|->]
  (m-1-1) edge node [above] {$\gamma$} (m-1-2)
          edge [bend left=40] node [auto] {$x\circ \gamma$} (m-1-3)
          edge node [auto] {$f\circ \gamma$} (m-2-2)
  (m-1-3) edge [bend left=15] node [auto] {$x^{-1}$} (m-1-2)
          edge node [right] {$(f\circ x^{-1})$ } (m-2-2)
  (m-1-2) edge node [left] {$f$} (m-2-2)
          edge node [auto] {$x$} (m-1-3);
\end{tikzpicture}

\[
\begin{gathered}
  (f\circ \gamma)'(\lambda) = (Df)\cdot \dot{\gamma}(\lambda) = \frac{ \partial f}{ \partial x^j} \dot{\gamma}^j(\lambda) = 2x (-\sin{\lambda} ) + 2y \cos{\lambda} = 2(-\cos{\lambda} \sin{\lambda} + \sin{\lambda} \cos{\lambda} ) = 0 
\end{gathered}
\]
