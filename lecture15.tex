\section{Einstein gravity}

Recall that in Newtonian spacetime, we were able to reformulate the Poisson law $\Delta \phi = 4\pi G_N \rho$ in terms of the Newtonian spacetime curvature as 
\[
R_{00} = 4\pi G_N \rho
\]
$R_{00}$ with respect to $\nabla_{\text{Newton}}$, and $G_N = $ Newtonian gravitational constant.

This prompted Einstein to postulate that the relativistic field equations for the Lorentzian metric $g$ of (relativistic) spacetime
\[
R_{ab} = 8\pi G_N T_{ab}
\]
However, this equation suffers from a problem. We know from matter theory that in RHS, $(\nabla_a T)^{ab} = 0$ since this has been formulated from an action. But in LHS, $(\nabla_a R)^{ab} \neq 0$ generically. Einstein tried to argue this problem away. Nevertheless, the equations cannot be upheld.

\subsection{Hilbert}
Hilbert was a specialist for variational principles. To find the appropriate LHS of the gravitational field equations, Hilbert suggested to start from an action
\[
S_{\text{Hilbert}}[g] = \int_M \sqrt{-g} R_{ab} g^{ab}
\]
which, in a sense, is formulated in terms of ``simplest action''. \\
\underline{Aim}: varying this w.r.t. metric $g_{ab}$ will result in some tensor $G^{ab}$.

\subsection{Variation of $S_{\text{Hilbert}}$}
\begin{align*}
  0 \overset{!}{=} \underbrace{\delta}_{g_i} S_{\text{Hilbert}}[g] = \int_M [ \underbrace{\delta \sqrt{-g} }_{1} \, g^{ab}R_{ab} + \sqrt{-g} \, \underbrace{\delta g^{ab}}_{2} R_{ab} + \sqrt{-g} \, g^{ab} \underbrace{\delta R_{ab}}_{3} ] 
\end{align*}

ad 1: $\delta \sqrt{-g} = \frac{- (\text{det}g)g^{mn} \delta g_{mn}}{2 \sqrt{-g}} = \frac{1}{2} \sqrt{-g} g^{mn} \delta g_{mn}$ \\
the above comes from $\delta \text{det}(g) = \text{det}(g) g^{mn} \delta g_{mn} \text{  e.g. from } \text{det}(g) = \exp{\text{tr}{\ln{g}}}$

ad 2: $g^{ab}g_{bc} = \delta^a_c \implies (\delta g^{ab})g_{bc} + g^{ab}(\delta g_{bc}) = 0  \implies \delta g^{ab} = -g^{am} g^{bn} \delta g_{mn}$

ad 3: \begin{align*}
\Delta R_{ab} & \underbrace{=}_{\text{normal coords at point}} \delta \partial_b \ccf{m}{am} - \delta \partial_m \ccf{m}{ab} + \Gamma \Gamma - \Gamma \Gamma \\
& = \partial_b \delta \ccf{m}{am} - \partial_m \delta \ccf{m}{ab} = \nabla_b (\delta \Gamma)\indices{^{m}_{am}} - \nabla_m (\delta \Gamma)\indices{^{m}_{ab}} \\
& \implies \sqrt{-g} g^{ab} \delta R_{ab} = \sqrt{-g}
\end{align*}
``if you formulate the variation properly, you'll see the variation $\delta$ commute with $\partial _b$''
%EY : 20150408 I think one uses the integration at the bounds, integration by parts trick

$\ccfx{i}{jk}{(x)} - \widetilde{\Gamma_{(x)}}\indices{^i_{jk}}$ are the components of a $(1,2)$-tensor. \\
Let us use the notation: $(\nabla_b A)\indices{^i_j} =: A\indices{^i_{j;b}}$

\[
\therefore \sqrt{-g} g^{ab} \delta R_{ab} \underbrace{=}_{ \nabla g = 0 } \sqrt{-g} (g^{ab} \delta \ccf{m}{am})_{;b} - \sqrt{-g} (g^{ab} \delta \ccf{m}{ab})_{ ; m} = \sqrt{-g} \, A\indices{^b_{;b}} - \sqrt{-g} \, B\indices{^m_{, m}}
\]

Question: Why is the difference of coefficients a tensor?

Answer:
\begin{align*}
\ccfx{i}{jk}{(y)} = \cibasis[y^i]{x^m} \cibasis[x^m]{y^j} \cibasis[x^q]{y^k} \ccfx{m}{nq}{(x)} + \cibasis[y^i]{x^m} \frac{ \partial^2 x^m}{ \partial y^j \partial y^k}
\end{align*}

Collecting terms, one obtains
\begin{align*}
  0 & \overset{!}{=} \delta S_{\text{Hilbert}} = \int_M [ \frac{1}{2} \sqrt{-g} \, g^{mn} \delta g_{mn} g^{ab} R_{ab} - \sqrt{-g} \, g^{am} g^{bn} \delta g_{mn} R_{ab} +    \underbrace{(\sqrt{-g} \, A^a)_{ \, , a} }_{\text{surface}} - \underbrace{(\sqrt{-g} \, B^b)_{ \, , b }}_{\text{surface term}}] \\
  & = \int_M \sqrt{-g} \, \delta \underbrace{g_{mn}}_{\text{arbitrary variation}} [\frac{1}{2} g^{mn} R - R^{mn}] \implies G^{mn} = R^{mn} - \frac{1}{2} g^{mn} R
\end{align*}

Hence Hilbert, from this ``mathematical'' argument, concluded that one may take
\[
\boxed{ R_{ab} - \frac{1}{2} g_{ab} R = 8 \pi G_N T_{ab} }  \\
\]
Einstein equations
\[
S_{E-H}[g] = \int_M \sqrt{-g} \, R
\]

\subsection{Solution of the $\nabla_a T^{ab} =0$ issue}
One can show ($\to$ Tutorials) that the \underline{Einstein curvature}
\[
G_{ab} = R_{ab} - \frac{1}{2} g_{ab}R
\]
satisfy the so-called \underline{contracted differential Bianchi identity} $(\nabla_a G)^{ab} = 0$.

\subsection{Variants of the field equations}
\begin{enumerate}[(a)]
\item a simple rewriting:
\begin{align*}
& R_{ab} - \frac{1}{2} g_{ab} R = 8 \pi G_N T_{ab} = T_{ab} && (G_N = \frac{1}{8\pi}) \\
& R_{ab} - \frac{1}{2} g_{ab} R = T_{ab} \, || \, g^{ab} && (\text{contract on both sides with } g^{ab}) \\
& R - 2R = T := T_{ab}g^{ab} \\
\implies & R = -T \\
\implies & R_{ab} + \frac{1}{2} g_{ab} T = T_{ab} \\
\Longleftrightarrow & R_{ab} = (T_{ab} - \frac{1}{2} Tg_{ab}) =: \widehat{T}_{ab} \\
\therefore \quad & \boxed{ R_{ab} = \widehat{T}_{ab}}
\end{align*}

\item $S_{E-H}[g] := \int_M \sqrt{-g} (R+ 2\Lambda)$ \quad \quad ($\Lambda$ is called cosmological constant)

\underline{History:} \\
1915: $\Lambda < 0$ (Einstein) in order to get a non-expanding universe \\
$>$1915: $\Lambda = 0$ (Hubble) \\
today: $\Lambda > 0$ to account for an accelerated expansion \\
$\Lambda \neq 0$ can be interpreted as a contribution $-\frac{1}{2} \Lambda g$ to the energy-momentum of matter in spacetime. This energy, which does not interact with anything but contributes to the curvature is called ``dark energy''.
\end{enumerate}

Question: surface terms scalar?

Answer: for a careful treatment of the surface terms which we discarded, see, e.g. E. Poisson, ``A relativist's toolkit'' C.U.P. ``excellent book''

Question: What is a constant on a manifold? \\
Answer: $\int \sqrt{-g} \, \Lambda = \Lambda \int \sqrt{-g} \, 1$

[back to dark energy]

[Weinberg used QCD to calculate $\Lambda$ using the idea that $\Lambda$ could arise as the vacuum energy of the standard model fields. It turns out that \\
$\Lambda_{\text{calculated}} = 10^{120} \times \Lambda_{\text{obs}}$ \\
which is called the ``worst prediction of physics''.


\underline{Tutorials}: \underline{check that }
\begin{itemize}
\item Schwarzscheld metric (1916)
\item FRW metric 
\item pp-wave metric 
\item Reisner-Nordstrom 
\end{itemize}
$\Longrightarrow $ are solutions to Einstein's equations
\end{enumerate}

in high school 

$m\ddot{x} + m\omega^2 x^2=0$

$x(t) = \cos{(\omega t)}$

\underline{ET}: [elementary tutorials]

study motion of particles \& observers in Schwarzschild S.T.

\underline{Satellite lectures}: \\
Marcus C. Werner: Gravitational lensing

odd number of pictures Morse theory (EY:20150408 Morse Theory !!!)

Domenico Giulini: Canonical Formulations of GR

Hamiltonian form

Key to Quantum Gravity
