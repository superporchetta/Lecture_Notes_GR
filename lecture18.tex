\section{L18: Canonical Formulation of GR-I}

\subsection{Dynamical and Hamiltonian formulation of General Relativity}
Purpose:
\begin{enumerate}[1)]
\item formulate and solve initial-value problems
\item integrate Einstein's Equations by numerical codes
\item characterise degrees of freedom
\item characterise isolated systems, associated symmetry groups and conserved quantities like Energy/Mass, Momenta (linear and angular), Poincare charges
\item starting point for ``canonical quantisation'' program
\end{enumerate}

How do we achieve this goal? We will rewrite Einstein's Equations in form of a constrained Hamiltonian system.

\[
\underbrace{R\indices{_{\mu\nu}} - \frac{1}{2} g\indices{_{\mu\nu}}R}_{G\indices{_{\mu\nu}}} + \underbrace{\Lambda}_{\text{cosmological constant}} g\indices{_{\mu\nu}} = \underbrace{k}_{\frac{8 \pi G}{c^4}} T\indices{_{\mu\nu}}
\]
$k = \frac{8 \pi G}{c^4}$ is an important quantity as it turns the energy density $T\indices{_{\mu\nu}}$ into curvature. \\
Physical dimensions: \\
\begin{align*}
\text{for curvature, } [G\indices{_{\mu\nu}}] & = \frac{1}{m^2}, \\
\text{for energy density } [T\indices{_{\mu\nu}}] & = \frac{\text{Joule}}{m^3} \\
\therefore \, [k] & = \frac{\frac{1}{m^2}}{\frac{J}{m^3}} = \frac{m}{J}
\end{align*}
 
\begin{framed}
\textbf{Convention} (for this lecture): \\
Greek indices run from $0$ to $3$ and latin indices from $1$ to $3$ \\
signature is $(-,+,+,+)$ as it makes space positive definite in $3+1$-decomposition\\
$T\indices{_{00}}$ is positive energy density. \\
\end{framed}
