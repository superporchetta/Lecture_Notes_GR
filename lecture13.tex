\section{Lecture 13: Relativistic spacetime}

Recall, from Lecture 9, the definition of Newtonian spacetime
\[
(M, \mathcal{O}, \mathcal{A}, \nabla, t) \quad \quad \quad \, \begin{aligned}
& \nabla \text{ torsion free } \\
& t \in C^{\infty}(M) \\ 
& dt \neq 0 \\
& \nabla dt = 0   \quad \, \text{ (uniform time) }
\end{aligned}
\]
and the definition of relativistic spacetime (before Lecture )


\[
(M, \mathcal{O}, \mathcal{A}^{\uparrow}, \nabla, g, T ) \quad \quad \quad \, \begin{aligned}
& \nabla \text{ torsion-free } \\
& g \text{ Lorentzian metric} (+---) \\ 
& T \text{ time-orientation }
\end{aligned}
\]

\subsection{Time orientation}

\begin{definition}
  $(M,\mathcal{O},\mathcal{A}^{\uparrow},g)$ a Lorentzian manifold.  Then a time-orientation is given by a vector field $T$ that 
\begin{enumerate}
\item[(i)] does \textbf{not} vanish anywhere 
\item[(ii)] $g(T,T)>0$
\end{enumerate}
\end{definition}

Newtonian vs. relativistic \\
Newtonian \\
$X$ was called future-directed if 
\[
dt(X) >0
\]
$\forall \, p \in M$, take half plane, half space of $T_pM$ \\
also stratified atlas so make planes of constant $t$ straight \\
relativistic \\
half cone $\forall \, p, q \in M$, half-cone $\subseteq T_pM$ \\

This definition of \underline{spacetime}

Question \\
I see how the cone structure arises from the new metric. I don't understand however, how the $T$, the time orientation, comes in \\

Answer \\
$(M,\mathcal{O}, \mathcal{A},g)$ $g \xleftarrow (+---)$

requiring $g(X,X)>0$, select cones \\
$T$ chooses which cone \\

This definition of \underline{spacetime} has been made to enable the following physical postulates:
\begin{enumerate}
\item[(P1)] The worldline $\gamma$ of a \underline{massive} particle satisfies
\begin{enumerate}
  \item[(i)] $g_{\gamma(\lambda)}(v_{\gamma, \gamma(lambda)} , v_{\gamma,\gamma(\lambda)} ) >0$
  \item[(ii)] $g_{\gamma(\lambda)}(T, v_{\gamma,\gamma(\lambda)}) >0$
\end{enumerate}
\item[(P2)] Worldlines of \underline{massless} particles satisfy
\begin{enumerate}
\item[(i)] $g_{\gamma(\lambda)}(v_{\gamma,\gamma(\lambda)}, v_{\gamma,\gamma(\lambda)}) = 0$
\item[(ii)] $g_{\gamma(\lambda)}(T,v_{\gamma,\gamma(\lambda)}) >0$
\end{enumerate}
\underline{picture}: spacetime:
\end{enumerate}

Answer (to a question) $T$ is a smooth vector field, $T$ determines future vs. past, ``general relativity: we have such a time orientation; smoothness makes it less arbitrary than it seems'' -FSchuller,


\underline{Claim}: $9/10$ of a metric are determined by the cone

spacetime determined by distribution, only one-tenth error 

\subsection{Observers} $(M,\mathcal{O}, \mathcal{A}^{\uparrow},\nabla ,g, T)$
\begin{definition}
  An \underline{observer} is a worldline $\gamma$ with
\[
\begin{aligned}
  & g(v_{\gamma}, v_{\gamma}) >  0 \\ 
  & g(T,v_{\gamma}) > 0 
\end{aligned}
\]
together with a choice of basis
\[
v_{\gamma,\gamma(\lambda)} \equiv e_0(\lambda) , e_1(\lambda), e_2(\lambda), e_3(\lambda)
\]
of each $T_{\gamma(\lambda)}M$ where the observer worldline passes, if $g(e_a(\lambda), e_b(\lambda)) = \eta_{ab} = \left[ \begin{matrix} 1 & & & \\ & -1 & & \\ & & -1 & \\ & & & -1 \end{matrix} \right]_{ab}$

\underline{precise}: observer $=$ \underline{smooth} curve in the frame bundle $LM$ over $M$
\end{definition}

\subsubsection{Two physical postulates}

\begin{enumerate}
  \item[(P3)] A \textbf{clock} carried by a specific observer $(\gamma, e)$ will measure a \textbf{time}
\[
\tau := \int_{\lambda_0}^{\lambda_1} d\lambda \sqrt{ g_{\gamma(\lambda)}(v_{\gamma,\gamma(\lambda)}, v_{\gamma,\gamma(\lambda)}) }
\]
between the two ``\underline{events}''
\[
\gamma(\lambda_0) \quad \quad \quad \, \text{ ``start the clock'' }
\]
and 
\[
\gamma(\lambda_1) \quad \quad \quad \, \text{ ``stop the clock'' }
\]
\underline{Compare} with Newtonian spacetime:
\[
t(p)=7
\]

Thought bubble: \underline{proper time/eigentime} $\tau$

\underline{Application/Example.}
$\begin{aligned}
& M = \mathbb{R}^4 \\ 
 & \mathcal{O} = \mathcal{O}_{\text{st}} \\
  & \mathcal{A} \ni (\mathbb{R}^4, \text{id}_{\mathbb{R}^4} ) \\ 
  & g : g_{(x)ij} = \eta_{ij} \quad \, ; \quad \quad \, T_{(x)}^i =(1,0,0,0)^i
\end{aligned}
$
\[
\Longrightarrow \Gamma_{(x) \, \, jk }^i = 0 \text{ everywhere }
\]
$\Longrightarrow (M,\mathcal{O}, \mathcal{A}^{\uparrow},g,T,\nabla)$ \quad \, $\text{Riemm}=0$ \\
$\Longrightarrow $ spacetime is flat

This situation is called special relativity.

Consider two observers: 
\[
\begin{aligned} & 
\begin{aligned}
& \gamma : (0,1) \to M \\ 
 & \gamma_{(x)}^i = (\lambda , 0 ,0  ,0 )^i \end{aligned} \\
& 
\begin{aligned}
  & \delta :(0,1) \to M \\
\alpha \in (0,1) :   & \delta_{(x)}^i = \begin{cases} ( \lambda , \alpha \lambda , 0 , 0)^i & \lambda \leq \frac{1}{2} \\ 
    (\lambda, (1-\lambda)\alpha, 0,0)^i & \lambda > \frac{1}{2} \end{cases}
\end{aligned}
\end{aligned}
\]
let's calculate:
\[
\begin{aligned}
  & \tau_{\gamma}:= \int_0^1 \sqrt{ g_{(x)ij} \dot{\gamma}^i_{(x)} \dot{\gamma}^j_{(x)} } = \int_0^1 d\lambda 1 = 1 \\
  & \tau_{\delta} := \int_0^{1/2} d\lambda \sqrt{ 1- \alpha^2} + \int_{1/2}^1 \sqrt{ 1^2 - (-\alpha)^2 } = \int_0^1 \sqrt{ 1 - \alpha^2 } = \sqrt{ 1 - \alpha^2}
\end{aligned}
\]
Note: piecewise integration

Taking the clock postulate (P3) seriously, one better come up with a realistic clock design that supports the postulate. 
\underline{idea}.

2 little mirrors
\item[(P4)] \underline{Postulate}

Let $(\gamma, e)$ be an observer, and \\
$\delta$ be a \emph{massive} particle worldline that is parametrized s.t. $g(v_{\gamma}, v_{\gamma})=1$ (for parametrization/normalization convenience)

Suppose the observer and the particle \emph{meet} somewhere (in spacetime)
\[
\delta(\tau_2) = p = \gamma(\tau_1)
\]

\emph{This} observer measures the 3-velocity (spatial velocity) of this particle as 
\begin{equation}\label{Eq:spatialv}
v_{\delta}: \epsilon^{\alpha}( v_{\delta, \delta(\tau_2)} ) e_{\alpha} \quad \quad \quad \, \alpha =1,2,3
\end{equation}
where $\epsilon^0, \boxed{ \epsilon^1,\epsilon^2,\epsilon^3}$ is the unique dual basis of $e_0,\boxed{ e_1,e_2,e_3}$
\end{enumerate}

EY:20150407

There might be a major correction to Eq. (\ref{Eq:spatialv}) from the Tutorial 14 : Relativistic spacetime, matter, and Gravitation, see the second exercise, Exercise 2, third question:
\begin{equation}
v := \frac{ \epsilon^{\alpha}({v}_{\delta} ) }{ \epsilon^0({v}_{\delta}) } e_{\alpha}
\end{equation}

\underline{Consequence}:
An observer $(\gamma, e)$ will extract quantities measurable in his laboratory from objective spacetime quantities always like that.

\underline{Ex}: $F$ Faraday $(0,2)$-tensor of electromagnetism:

\[
F(e_a,e_b) = F_{ab} = \left[ \begin{matrix} 0 & E_1 & E_2 & E_3 \\ 
    -E_1 & 0 & B_3 & -B_2 \\ 
    -E_2 & -B_3 & 0 & B_1 \\
    -E_3 & B_2 & -B_1 & 0 \end{matrix} \right]
\]
observer frame $e_a,e_b$

$E_{\alpha} := F(e_0,e_{\alpha})$ \\
$B^{\gamma}:= F(e_{\alpha},e_{\rho})\epsilon^{\alpha \beta \gamma}$
where 
$\epsilon^{123} = +1$ totally antisymmetric

\subsection{Role of the Lorentz transformations}

Lorentz transformations emerge as follows: \\
Let $(\gamma,e)$ and $(\widetilde{\gamma},\widetilde{e})$ be observers with $\gamma(\tau_1) = \widetilde{\gamma}(\tau_2)$

(for simplicity $\gamma(0) = \widetilde{\gamma}(0)$

Now 
\[
\begin{gathered}
  e_0 , \dots , e_1 \quad \quad \quad \, \text{ at } \tau = 0 \\
  \text{ and } 
  \widetilde{e}_0 , \dots , \widetilde{e}_1 \quad \quad \quad \, \text{ at }  \tau = 0 \\
\end{gathered}
\]
both bases for the same $T_{\gamma(0)}M$

\underline{Thus}: $\widetilde{e}_a = \Lambda^b_{ \, \, a} e_b $ \quad \quad \, $\Lambda \in GL(4)$

Now:

\[
\begin{aligned}
  \eta_{ab} = g(\widetilde{e}_a, \widetilde{e}_b) & = g(\Lambda^m_{ \, \, a}e_m, \Lambda^n_{ \, \, b} e_n ) = \\
  & = \Lambda^m_{ \, \, a} \Lambda^n_{ \, \, b} \underbrace{g(e_m,e_n)}_{ \eta_{mn}}
\end{aligned}
\]
i.e. $\Lambda \in O(1,3)$

\underline{Result}: Lorentz transformations relate the \emph{frames} of \emph{any two observers} at the same point.

``$\widetilde{x}^{\mu} - \Lambda^{\mu}_{ \, \, \nu} x^{\nu}$'' is utter nonsense

\subsection*{Tutorial}

I didn't see a tutorial video for this lecture, but I saw that the Tutorial sheet number 14 had the relevant topics.  Go there.

